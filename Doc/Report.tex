\documentclass[12pt, a4paper]{article}
\usepackage[UTF8]{ctex}
\usepackage{graphicx}
\usepackage{caption2}
\usepackage{subfigure}
\usepackage{colortbl,booktabs}
\usepackage{appendix}
\usepackage{geometry}
\geometry{left = 2.5cm, right = 2.5cm, top = 2.5cm, bottom = 2.5cm}

% Title
\title{\textbf{\LARGE 2019年夏季学期 \quad 程序设计训练 \\ 第一周大作业报告}}
\author{\textbf{魏彤} \\ 计86班 \quad 2018011417}
\date{2019年8月24日}
\renewcommand{\thefootnote}{\fnsymbol{footnote}}

\begin{document}
	
	\maketitle
	
	\begin{abstract}
		本次大作业基于Qt实现了数字微流控生物芯片的模拟界面。用户输入指令文件、设置参数后,程序能够以图形化界面展示液滴在芯片上运动的动态过程,并播放相应的仿真音效。除此之外,对于数字微流控生物芯片中的污染问题,程序实现了污染状况实时绘制和简易的清洗液滴路径算法设计,避免不同种类的液滴污染的情况发生。
	\end{abstract}
	
	\section{图形界面设计}
		图形界面包括程序主界面、设置芯片参数和清洗液滴参数的界面和弹窗。主要使用Qt Creator进行设计,辅以代码进行调整和补充。\footnote{界面图片见附录A,均为在Mac系统中显示效果。}
		\subsection{程序主界面}
		程序主界面包括菜单栏\&工具栏、芯片绘制区、指令显示区和时间显示区。 \\
		
		菜单栏和工具栏设置了所有指令的按钮,并设置了快捷键。
		指令显示区在输入指令文件后显示文件内容。时间显示区显示模拟时的时间。
		\begin{table}  
			\centering  
			\caption{菜单栏工具栏指令}  
			\begin{tabular}  
				{cccccc}  
				\toprule[1pt]  
				指令 & 功能 & 快捷键(表中显示Mac键盘)  \\  
				\midrule  
				New Simulation    & 打开芯片参数设置界面      & command + N \\  
				Washer   	      & 打开清洗液参数设置界面    & command + W \\
				Step Previous     & 向前步进一步   	       & 左 \\
				Step Next	      & 向后步进一步			   & 右 \\
				Play All          & 播放至最后				& option + 右 \\
				Reset             & 重置				     & option + 左  \\
				Inspect Pollution & 切换液滴界面/污染数界面   & command + P \\
				\bottomrule[1pt]  
			\end{tabular}  
		\end{table}
		芯片绘制区根据当前状态绘制芯片,包括液滴出入口、液滴、污渍、清洗液出入口,在切换后可以显示每个电极的污染数,还可以通过鼠标右键点击设置清洗液障碍。
		
		
	
		\subsection{参数界面}
		设置芯片参数界面中,芯片的行数列数和出口位置由数字框输入,入口位置通过设置输入栏,用户可以添加删除条目。\\ \hspace*{0.8cm}		
		设置清洗液参数界面中,勾选框可以设置是否开启清洗功能,数字框中能够输入清洗液出入口的位置。
		
	
	
		\subsection{弹窗界面}
		利用Qt中内置的QMessagebox类进行代码实现,用于当出现错误时进行提示。
		
		\section{代码架构设计}
		大作业项目采用OOP的设计思路,除main.cpp文件外,共包含以下几个部分。
		\begin{itemize}
			\item \textbf{MainWindow} \\ \hspace*{0.8cm}	主界面对应的类,为所有部件的父对象。负责主界面的显示和不同部件之间的通讯功能。
			\item \textbf{chip} \\ \hspace*{0.8cm}	芯片绘制区对应的类,核心功能的实现区域。负责芯片状态的描述、绘制和操作。
			\item \textbf{fileManager} \\ \hspace*{0.8cm}	文件系统对应的类。负责文件的读入、指令的解析。
			\item \textbf{command} \\ \hspace*{0.8cm}	指令对应的类,负责描述基本操作的内容。
			\item \textbf{Dialog} \\ \hspace*{0.8cm}	芯片参数设置界面对应的类。负责相应界面的交互和数据的传输。
			\item \textbf{\large washDialog} \\ \hspace*{0.8cm}	清洗液参数设置界面对应的类。负责相应界面的交互和数据的传输。
			\item \textbf{\large waterDrop} \\ \hspace*{0.8cm}	液滴对应的类,负责存储每个液滴对应的信息。
			\item \textbf{\large stainCommand} \\ \hspace*{0.8cm}	污渍更改指令对应的类,负责存储每一次污渍产生或修改的记录。
		\end{itemize}
		
		\section{功能设计}
		针对需求,程序实现了以下七项基本功能。
		\subsection{基本的界面设置和显示}
			MainWindow类继承自QMainWindow,包含了菜单栏和工具栏。每个指令都通过QAction类进行内容、图标、快捷键、信号的设定,之后加入到ui的menu和toolbar中。 \\ \hspace*{0.8cm}
			点击New Simulation按钮后,参数设置界面弹出。界面的每个输入部件都通过connect函数与Dialog类中相应的信号槽相连接,改变Dialog类中储存的相应信息。\\ \hspace*{0.8cm}
			Dialog类对输入进行条件判断(包括行列数不同时小于3,以及输入输出口在芯片边缘),如果不符合则报错,否则关闭界面并将数据以信号-信号槽的方式传给chip类,chip类中的相应信息进行更新,并开始在画布上按照信息绘制芯片。 \\ \hspace*{0.8cm}
			画布是一个白色背景的QWidget,大小固定。根据行列数,确定每一格电极的大小。利用Qt中的QPainter类中的drawRect()函数绘制芯片和输入输出端口。
			
		\subsection{文件输入与指令处理}
			此功能在fileManager类中实现,利用QFile类中的getOpenFileName()函数选择文件,打开后将指令一行行读入至QStringList中。 \\ \hspace*{0.8cm}
			遍历QStringList的指令,按照空格和逗号切分,得到指令的类型、时间和参数,推入command类的multiset中进行存储并传输给chip类。值得注意的是,command类的小于号重载为时间关键字比较,故multiset可以将所有指令按照时间顺序排列好,为后续操作提供便利。 \\ \hspace*{0.8cm}
			通过观察操作的特性和联系,将所有的指令都归为七种,时间为1单位的基本指令:Move、Input、Output、Split1(拉伸)、Split2(分离)、Merge1(合并)、Merge2(压缩),为后续操作的执行做基础。
			
		\subsection{模拟过程的双向实现}
			chip类中的operation()和operationReverse()两个函数分别针对指令的类型实现对应的操作。chip类中的water是一个存储液滴的set,根据操作的参数,在其中找到对应的液滴,根据操作内容生成新的液滴,然后在set中替换,从而完成操作。\\ \hspace*{0.8cm}
			注意到每个基本操作的逆操作都是另外一个基本操作。Move - Move; Input - Output; Split1 - Merge2; Split2 - Merge1。因此,通过每个逆向操作与对应的正向操作类似,从而更加易于实现。\\ \hspace*{0.8cm}
			利用QTime创造了一个计时器,当Play All被点击时,计时器被打开,以500ms的间隔发射信号,触发操作下一步指令的信号槽,从而实现了播放的功能。计时器直到最后一个操作被执行或者Reset按钮被按下才会终止 \\ \hspace*{0.8cm}
			Reset按钮按下后信号触发函数,将所有chip中的成员均重置为初始化的值。
		\subsection{播放音效的实现}
			每个操作对应的音效利用QSoundEffect库进行实现。将所有的音频文件封装在.qrc文件中,chip类初始化时建立QSoundEffect指针,在operate()函数中每个基本操作开始的地方进行调用和播放。
		\subsection{液滴移动的约束检查}
			
		
		\newpage
		\begin{appendices}
			\section{图形界面图片}
			
			\begin{figure}[htbp]
				\centering
				\includegraphics[width=15cm]{Img/menu.png}
				\caption{菜单栏}
			\end{figure}
		
			\begin{figure}[htbp]
				\centering
				\includegraphics[width=15cm]{Img/main.png}
				\caption{主界面}
			\end{figure}
		
			\begin{figure}[htbp]
				\centering
				\includegraphics[width=15cm]{Img/init.png}
				\caption{芯片参数设置界面}
			\end{figure}
		
			\begin{figure}[htbp]
				\centering
				\includegraphics[width=5cm]{Img/washer.png}
				\caption{清洗液参数设置界面}
			\end{figure}
		\end{appendices}
\end{document} 
